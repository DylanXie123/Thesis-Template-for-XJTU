\documentclass[UTF8,bachelor]{XJTUthesis}

\begin{document}

\firstpage{智能医疗机器人云存储技术研究及}{交互系统设计}{钱学森学院}{机械自动化}{机械钱31}{靳宇栋}{2161700000}{许睦旬}{西安交通大学}

\tableofcontents
\thispagestyle{empty}
\setcounter{page}{0}
\newpage

\begin{abstract}
医疗资源紧张和医疗服务滞后是当今医疗领域的突出问题,医疗机器人可以打破传统医疗服务的时间和空间限制,缓解医护人员的繁忙机械劳动。论文在一种新型服务机器人基础上,规划了全面的机器人内容服务体系,对以机器人为媒介的远程医疗模式进行了研究。首先,基于Android平台开发了智能医疗机器人交互系统:通过蓝牙BLE实现了体征数据无线采集,基于HelloChart实现了体征数据折线图的动态绘制。应用云技术实现了康复进度反馈、智能医嘱和远程随访功能、云端药品信息库的远程管理。基于SDK搭建了用户管理系统,增强了产品化的用户体验。其次,实现了机器人底层功能与内容服务的对接:应用WebSocket技术搭建了交互系统与主控系统之间的通讯系统,并基于JSON制定了两系统之间的数据传输协议。最后,基于Bmob搭建了其云存储平台:实现了结构化和非结构化数据的云端统一管理,实现了云端和本地的数据实时同步。通过样机调试,整套智能医疗机器人交互系统功能丰富、界面美观、操作友好,其云存储平台管理有序、运行稳定,对新型智能医疗机器人产品的最终开发定型起到支撑作用,具有较高的应用价值。
\end{abstract}
\keywords{医疗服务;智能医疗机器人;云存储技术;交互系统设计}
\newpage
\begin{eabstract}
Times New RomanTimes New RomanTimes New RomanTimes New RomanTimes New RomanTimes New RomanTimes New RomanTimes New RomanTimes New RomanTimes New RomanTimes New RomanTimes New RomanTimes New RomanTimes New Roman\par
Times New RomanTimes New RomanTimes New RomanTimes New RomanTimes New RomanTimes New RomanTimes New RomanTimes New RomanTimes New RomanTimes New RomanTimes New RomanTimes New RomanTimes New RomanTimes New Roman\par
Times New RomanTimes New RomanTimes New RomanTimes New RomanTimes New RomanTimes New RomanTimes New RomanTimes New RomanTimes New RomanTimes New RomanTimes New RomanTimes New RomanTimes New RomanTimes New Roman\par
\end{eabstract}
\ekeywords{Good;Doo}

\section{绪论}
随着人口老龄化和生活质量不断提高,人们对医疗服务的数量和质量都提出了更高的要求。如今,基于机器人技术、云存储技术的远程医疗服务模式的研究工作愈加深入,但具体到远程医疗机器人的研究应用,还没有上升到产品化的高度。本智能医疗机器人主要面向患者术后康复治疗、慢性病患者、护理院三个应用场景,旨在实现对医疗优势智力资源的共享,创建以机器人为媒介的新型医疗服务模式,论文主要开展以下工作:\par
开发人性化的交互系统,在其中集成全面的内容服务模块,实现提供便捷医疗服务。构建交互系统与主控系统的数据交互机制,将内容服务与自主导航等机器人技术进行融合,实现机器人化的医疗服务。搭建云存储平台,将机器人运行过程中涉及的所有动态数据部署在云端,实现医患之间、各机器人之间的数据实时同步,构建畅通的医患交互桥梁。\par

\section{智能医疗机器人系统架构}
智能医疗机器人整体结构如图\ref{overall_struct}所示,其中共有主控系统、交互系统、云平台三个控制端。
\begin{figure}[htbp]
  \centering
  \includegraphics[width=0.5\textwidth]{example//overallstruct.png}
  \caption{智能医疗机器人整体结构}\label{overall_struct}
\end{figure}
主控系统通过协调控制语音、激光雷达、导航底盘等技术模块,实现机器人的拟人化功能输出;交互系统通过人机交互界面提供医疗服务,并实现机器人的可视化控制;云存储平台是机器人的数据管理中枢,主要负责动态数据的存储、管理与分析。智能医疗机器人各控制端的定位与运行机制如图\ref{flowchart}所示。\par
\begin{figure}[htbp]
  \centering
  \includegraphics[width=0.8\textwidth]{example//flowchart.png}
  \caption{各控制端的定位与运行机制}\label{flowchart}
\end{figure}
智能医疗机器人软件系统架构如图2-3所示。基于ROS机器人操作系统开发的机器人控制端,可统一处理各功能硬件的数据,并将其转化为机器人行为输出。采用Android平台开发的交互系统控制端,首先可保持与云存储平台之间的实时数据通信,其次保持与主控制端之间的数据通信,从而控制机器人进行正确的行为输出。

\section{交互系统设计及实现}
\subsection{需求分析与架构设计}
根据应用场景和设计目标,智能医疗机器人的需求归纳如下:在功能方面,划分为体征监测、远程随访、智能医嘱、康复进度反馈、机器人控制、用户管理等模块;在数据管理方面,需保证所有云端数据的本地实时更新;在界面设计方面,交互系统界面架构要合理划分,在视觉体验上要具有时代感。根据需求,选择了Android作为交互系统开发平台。进行内容服务模块页面的重组划分,如图\ref{module}所示。
\begin{figure}[htbp]
  \centering
  \includegraphics[width=0.5\textwidth]{example//module.png}
  \caption{内容服务模块划分}\label{module}
\end{figure}
交互系统的项目架构部署如图3-2所示。除了编译所需的配置与相关代码编写,主要关注前端静态界面res和后端功能实现java两部分内容。\par
\subsection{交互系统UI设计}
通过人机学研究,最终选择了\# 0195ff道奇蓝、\# 404040深灰、\# f0f0f6淡灰、\# fffafa雪白作为主题色,并设计了机器人产品LOGO与部分素材,如图3-3所示。\par
由于Android官方库有限,因此编写制作了滑动选择器、底部弹出式选择框、确认与取消提示框等UI控件。部分控件如图3-4所示。

\subsection{界面设计与功能实现}
\subsubsection{主页面模块}
主页面与机器人形象动态页面如图3-5所示。主页面采用了底栏和顶栏的双级目录结构,旨在使用户可以通过少于两步操作到达任何一个功能模块。通过顶栏右方的人形按键可以跳转至用户管理页面。为了使交互系统机器人化,实现操作静止20秒后会自动跳转机器人卡通形象,点击屏幕可以跳转回主页面。
\subsubsection{症状反应模块}
症状反应信息是医生对于患者康复情况判断的重要依据,该模块以选择题问卷的形式呈现,配合机器人周期性的智能提醒,患者可以方便且无遗漏地反馈症状情况;医生可实时查看患者症状反应信息统计,及时修改治疗方案,相关界面如图3-6所示。
\subsubsection{生命体征模块}
传统的体征数据反馈方式往往是通过医生向患者的问询,这种反馈模式容易造成信息有误或丢失。该模块旨在帮助医生对患者形成有效的健康体征数据的远程监控。该模块界面如图3-8所示。
\subsubsection{常用工具封装}
“封装性”是JAVA语言的重要特点。本节实现了SharedPreference存储工具、缓存清理工具、文本格式判断工具的封装。缓存清理工具封装流程如表\ref{fuhao}所示。
\begin{table}[htbp]
  \centering
\begin{tabular}{|c|c|}
  \hline
  \makecell{符号}&\makecell{说明}\\ %makecell用于生成表头
  \hline
  $a_t$ & 参数在t时刻的值 \\
  \cline{1-2}
  $\rho$ & 相关系数 \\
  \cline{1-2}
  $\boldsymbol{\omega}$ & 超平面方向向量 \\
  \hline
\end{tabular}
  \caption{符号说明}\label{fuhao}
\end{table}

\section{交互系统与主控系统信息通讯实现}
“封装性”是JAVA语言的重要特点。本节实现了SharedPreference存储工具、缓存清理工具、文本格式判断工具的封装。缓存清理工具封装流程如表\ref{haha}所示。
\begin{table}[htbp]
  \centering
\begin{tabular}{|c|c|}
  \hline
  \makecell{符号}&\makecell{说明}\\ %makecell用于生成表头
  \hline
  $a_t$ & 参数在t时刻的值 \\
  \cline{1-2}
  $\rho$ & 相关系数 \\
  \cline{1-2}
  $\boldsymbol{\omega}$ & 超平面方向向量 \\
  \cline{1-2}
   $b$ & 超平面偏置量 \\
  \cline{1-2}
  \hline
\end{tabular}
  \caption{符号说明}\label{haha}
\end{table}
\end{document} 