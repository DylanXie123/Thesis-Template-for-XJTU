\documentclass{beamer}

\usepackage{ctex}
\useoutertheme[footline=authorinstitutetitle]{miniframes}
\useinnertheme{rectangles}
\usecolortheme{whale}
%\usecolortheme{crane}

\newcommand{\frameofframes}{/}
\newcommand{\setframeofframes}[1]{\renewcommand{\frameofframes}{#1}}
\setbeamertemplate{footline} 
{%
  \begin{beamercolorbox}[colsep=1.5pt]{upper separation line foot}
  \end{beamercolorbox}
  \begin{beamercolorbox}[ht=2.5ex,dp=1.125ex,%
    leftskip=.3cm,rightskip=.3cm plus1fil]{author in head/foot}%
    \leavevmode{\usebeamerfont{author in head/foot}\insertshortauthor}%
    \hfill%
    {\usebeamerfont{institute in head/foot}\usebeamercolor[fg]{institute in head/foot}\insertshortinstitute}%
  \end{beamercolorbox}%
  \begin{beamercolorbox}[ht=2.5ex,dp=1.125ex,%
    leftskip=.3cm,rightskip=.3cm plus1fil]{title in head/foot}%
    {\usebeamerfont{title in head/foot}\insertshorttitle}%
    \hfill%
    {\usebeamerfont{frame number}\usebeamercolor[fg]{frame number}\insertframenumber~\frameofframes~\inserttotalframenumber}
  \end{beamercolorbox}%
  \begin{beamercolorbox}[colsep=1.5pt]{lower separation line foot}
  \end{beamercolorbox}
}

\xdefinecolor{xjtu}{RGB}{0,78,151}
\xdefinecolor{xjtubak}{RGB}{0,46,230}

\setbeamercolor{structure}{bg=xjtu}
\setbeamercolor{title}{fg=white,bg=xjtu}
\setbeamercolor{frametitle}{fg=white,bg=xjtu}
\setbeamercolor{block title}{fg=white,bg=xjtubak}


\begin{document}

\title{TEST}
\author{DX}
\institute{大学}
\begin{frame}
  \titlepage
\end{frame}

\begin{frame}
  \frametitle{Outline}
  \tableofcontents %  [pausesections]
\end{frame}

\section{first section}
\begin{frame}
  \frametitle{1}
  \begin{itemize}
    \item Your answer is \usebeamertemplate{answer}.
    \item hello world!
  \end{itemize}
\end{frame}

\section{second section}
\subsection{first subsection}
\begin{frame}
  \frametitle{2.1}
  \begin{block}{碰撞检测问题}
    计算机图形学、虚拟现实等领域中的研究热点,
    是计算机模拟真实环境中不可或缺的技术,
    在物理仿真及游戏领域里应用十分广泛。
  \end{block}
  \note{
    凸包围体技术在计算机图形学领域里的各种算法中发挥着重要作用,如优化渲染和建模过程,加速求交、碰撞检测等算法。
    主要是用在原始模型之间的相关计算(遮挡测试、相交测试等)之前进行预处理判断和剪枝,以求交算法为例,如果两个模型相交,则对应的凸包围体一定相交,
    若凸包围体不相交则其对应的原始模型一定不相交。而一般来讲,判断凸包围体是否相交比判断原始模型相交更简单,因此可以提升效率。
    
    碰撞检测问题是计算机图形学、虚拟现实等领域中的研究热点,是计算机模拟真实环境中不可或缺的技术,在物理仿真及游戏领域里应用十分广泛。
    例如在游戏中,碰撞检测技术增强了游戏的真实性,游戏中的角色行走不可穿墙、角色中弹而亡等等都离不开碰撞检测技术。
  }
\end{frame}

\subsection{second subsection}
\begin{frame}
  \frametitle{2.2}
  test information
\end{frame}

\section{third section}
\begin{frame}
  \frametitle{3}
  test information
\end{frame}

\end{document} 