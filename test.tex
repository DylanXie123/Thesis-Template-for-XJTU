\documentclass[bachelor]{XJTUthesis}

%\addbibresource[location=local]{reference//example.bib}

\begin{document}

\firstpage{西安交通大学}{\LaTeX 毕业设计模板}{电气学院}{电气工程}{电气613}{谢晋安}{0000000000}{谢晋安}{西安交通大学}

\tableofcontents
\thispagestyle{empty}
\setcounter{page}{0}
\newpage

\begin{abstract}
这是一个模板。
\end{abstract}
\keywords{\LaTeX;XJTU}
\newpage
\begin{eabstract}
This is a template.
\end{eabstract}
\ekeywords{\LaTeX;XJTU}

\chapter{图表}
图\ref{xiaohui:red}和\ref{xiaohui:blue}是交大的校徽。
\begin{figure}[htbp]
  \centering
  \includegraphics[width=0.3\textwidth]{figures//a3_1jdxhred.png}
  \caption{校徽}\label{xiaohui:red}
\end{figure}

\begin{figure}[htbp]
  \centering
  \includegraphics[width=0.3\textwidth]{figures//a3_2jdxhblue.png}
  \caption{校徽}\label{xiaohui:blue}
\end{figure}

表\ref{fuhao}是符号说明
\begin{table}[htbp]
    \centering
    \begin{tabular}{|c|c|}
        \hline
        \makecell{符号}&\makecell{说明}\\ %makecell用于生成表头
        \hline
        $a_t$ & 参数在t时刻的值 \\
        \hline
        $\rho$ & 相关系数 \\
        \hline
        $\boldsymbol{\omega}$ & 超平面方向向量 \\
        \hline
        $b$ & 超平面偏置量 \\
        \hline
    \end{tabular}
    \caption{符号说明}\label{fuhao}
\end{table}

\chapter{图表2}
图\ref{xioabioa:shu}是竖版校标
\begin{figure}[htbp]
  \centering
  \includegraphics[width=0.5\textwidth]{figures//a4_7sbzh.png}
  \caption{校标}\label{xioabioa:shu}
\end{figure}

表\ref{tab:test}是常用的三线表
\begin{table}[htbp]
    \centering
    \begin{tabular}{lcl}
        \toprule
        。。 & 。。 & 。。 \\
        \midrule
        。。 & 。。 & 。。 \\
        。。 & 。。 & 。。 \\
        。。 & 。。 & 。。 \\
        \bottomrule
    \end{tabular}
    \caption{\label{tab:test}示例表格}
\end{table}

\chapter{TIKZ}

\chapter{一些环境}
algorithm环境
\begin{algorithm}
    \caption{Calculate $y = x^n$}
    \label{alg1}
    \begin{algorithmic}
        \REQUIRE $n \geq 0 \vee x \neq 0$
        \ENSURE $y = x^n$
        \STATE $y \leftarrow 1$
        \IF{$n < 0$}
        \STATE $X \leftarrow 1 / x$
        \STATE $N \leftarrow -n$
        \ELSE
        \STATE $X \leftarrow x$
        \STATE $N \leftarrow n$
        \ENDIF
        \WHILE{$N \neq 0$}
        \IF{$N$ is even}
        \STATE $X \leftarrow X \times X$
        \STATE $N \leftarrow N / 2$
        \ELSE[$N$ is odd]
        \STATE $y \leftarrow y \times X$
        \STATE $N \leftarrow N - 1$
        \ENDIF
        \ENDWHILE
    \end{algorithmic}
\end{algorithm}

lstlisting环境用于插入代码
\begin{lstlisting}[language=c++]
//hello.c
#include<stdio.h>
int main(void)
{
    int *p;
    printf("hello");
    return 0;
}
\end{lstlisting}

\begin{lstlisting}[language=matlab]
for i=1:100
    display('hello');
end
\end{lstlisting}

\chapter{杂项}

\begin{appendixs}
战斗风格十色入规范司法和v角色发v就是
\end{appendixs}

\begin{appendixs}
战斗风格十色入规范司法和v角色发v就是
\end{appendixs}

\begin{acknowledgement}
\chaptername
\end{acknowledgement}

%\printbibliography[title=参考文献]
\end{document} 